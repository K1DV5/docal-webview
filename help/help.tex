% -beta
\documentclass[12pt]{article}
\usepackage[colorlinks]{hyperref}
\usepackage{booktabs, graphicx}
\usepackage[margin=1.2in,a4paper] {geometry}
\usepackage{amsmath}

\begin{document}
\title{docal Help}
\author{K1DV5}
\maketitle

\tableofcontents

\section{Introduction}
\textbf{docal} lets you write what the procedure is supposed to be and takes
care of the rest of the work like calculating the numbers, formatting them in
scientific notation if needed, show the three steps of a calculation as
necessary and output a beautiful content to a document file of your choice.

\autoref{demo} provides a good introduction of what docal does. You write the
input document and the calculations and docal produces the output document.

\begin{figure} [ht]
    \includegraphics[width=0.7\textwidth]{demo.pdf}
    \caption{Demo}
    \label{demo}
\end{figure}

You leave the places
for the calculations by giving ``tags'' using the hashtag (from social
networks) notation: \#[tagname] where [tagname] is a unique name for that place
that the calculations will go.

Then in the calculations, you refer to the places using the [tagname] preceding
them with a single \# and giving the [tagname]. For example,

\begin{verbatim}
#sometag
x = 10
y = x^2
\end{verbatim}

will go to the place specified as \#sometag in the document.  When you're happy
with the preview, click ``Send''. docal produces the output document inserting
formatted, detailed calculations into the places of the tags.

You can create a new document from docal but your word processing options such
as creating figures, tables, headings, formatting paragraphs, etc. will be
absent as those are not docal's intended use. Thus, for big documents, it is
recommended to create the other parts of the document in the dedicated program
and use tags for the calculations. But you can use only docal for simple
worksheets.

\section{Usage}

The supported document types are Word and \LaTeX.

Generally docal works with up to 3 files.
\begin{description}
    \setlength\itemsep{0pt}
    \item[Input document:] A document file (Word or \LaTeX) in which you have
        written the other content and put tags in the desired places
    \item[Calculations file:] A file in which you write the calculations. It
        can be saved, opened etc.
    \item[Output document:] A document file that will be created by inserting
        the calculations into the document file
\end{description}

Not all of them need to be given (see \nameref{overview}). But generally, you specify the
input document name (the one containing the tags) and the output document name
(the one you want as combined, final document). Then you write the calculations
or optionally open a previously saved calculations file (you can save them.)
The press `Send'. A document will be created with the output name that you gave.
You can open both documents from within docal using the `Open' menu next to the
names.

While you write, you can insert options (see \nameref{options}) by typing them
directly or using the options sidebar on the right.

\subsection{Overview}
\label{overview}

The following are some points to consider.
\begin{itemize}
    \setlength\itemsep{0pt}
    \item The input and output documents cannot be of different filetypes: If
        the input is Word, the output must be Word.
    \item If you leave the input document name blank, and the output document
        name is given, docal will create a new document containing the
        calculations with no additional content.
    \item If you give the input document name and leave the output document
        name blank, docal will insert the calculations into the document and
        \begin{description}
            \setlength\itemsep{0pt}
            \item{For Word:} will save the resulting file by the name: [input
                file name]-out.docx. If a file by that name already exists, it
                overwrites it. \textbf{Be careful}.
            \item{For \LaTeX:} will save the resulting file by the same name,
                with the possibility of reversing (using the ``Clear tags'')
                menu item.
        \end{description}
    \item If the two document names are the same,
        \begin{description}
            \setlength\itemsep{0pt}
            \item{For Word:} The original file will be overwritten and the tag
                places will be replaced by calculations. This may not be
                desired as it will make rearranging the document more difficult
                because many equations make Word slow, and the tags will be
                lost. So \textbf{be careful}.
            \item{For \LaTeX:} The original will be overwritten but the
                operation is reversible (see above).
        \end{description}
    \item If there are more tags in the document than in the calculations, only
        those in the calculations will be processed and the rest will be
        ignored. If there are more tags in the calculations than in the
        document, only those that are in the document will be processed and the
        rest will be ignored. So try to keep a one to one correspondence for
        best results.
    \item If there is no tag in the document, the output document will be the
        same as the input document.
    \item If there are tags in the document, the first tag is implied in the
        calculations, so if you only want to use one tag, you don't need to
        specify it in the calculations. But you have to specify it in the
        document to tell it where the calculations will go.
    \item If you want to insert a \# character without confusing it for a tag
        beginning, precede it with a backslash (\verb|\|), e.g.
        \verb|\#twitter|
    \item If there were log messages produced when you click `Send', a they
        will appear in a message box. You can set the minimum log level to show
        using the select box above the `Send' button.
\end{itemize}

\section{Calculations content}

docal goes over the calculations line by line and the lines can be any of:

\begin{itemize}
    \setlength\itemsep{0pt}
    \item Tags e.g. \verb|#sometag|
    \item Calculated equations e.g. \verb|x = 5*y|
    \item Calculated expressions, e.g. \verb|5*x| or \verb|1/5 + 8|
    \item Text, eg. \verb|# The value of x|
    \item Equations and expressions just for display, e.g. \verb|#$$ x = 5*y|
    \item Comments (text that will not appear on the document) e.g.
        \verb|## some comment|
\end{itemize}

\subsection{Tags}

Tags are placeholders in the document and kind of titles in the calculations,
where anything below them is supposed to be inserted in place of them in the
document.  For example, take the following calculations:

\begin{verbatim}
#first
x = 5
y = 5*x
#second
z = x^2
d = 3
\end{verbatim}

Now the two calculations under \#first will go to that tag in the document, and
the other two under \#second will go to the place specified by \#second in the
document.

The tags can only be made up of:
\begin{itemize}
\setlength\itemsep{0pt}
    \item The \# character to begin
    \item No space between the \# character and the name or in between
    \item Any combination of letters, the \_ character, and numbers. So
        \#First, \#seCOnd2, \#step\_1 are valid but \# first, \#fi;rst are not
        valid.
\end{itemize}

\subsection{Calculated Equations}

These are the main contents. They are equations where the left hand side is a
variable name and the right hand side is an expression.  The rules come from
the underlying Python and are simplified as follow.

\begin{itemize}
    \setlength\itemsep{0pt}
    \item There must be an equal sign between the right and left hand sides. It
        cannot be other signs as $>, \leq$ etc. While this will not cause any
        errors, the result might not be the intended one as it will be
        evaluated as just an expression (see \nameref{expressions}).
    \item The left hand side cannot be an expression like $x + 1 = 5 \times
        10$. It can be a variable name. The variable name can only start with a
        letter or the \_ character and the rest can also contain numbers. So
        \verb|x_1, height, s1| are valid but \verb|9x, the height, 4 width| are
        not valid. It can also be more than one variable but there have to be
        corresponding values on the right side like \verb|x, y, z = 5, 8, 9|.
    \item every operation must be explicitly written, there is nothing implied.
        For example, to multiply 9 by x, we might be used to writing $9x$ but
        this is invalid here. Write \verb|9*x| instead. Also, to use functions
        like sin, sqrt, etc., enclose the values in brackets. So \verb|sin x|
        is invalid but \verb|sin(x)| is valid.
\end{itemize}

\subsubsection{Variable Name Customization}

Variable names can be extensively customized. You are not stuck with just
letters and numbers like $x$.

\begin{description}
    \setlength\itemsep{0pt}
    \item[Subscript, Superscript] These are widely used notations. To write a
        subscript to a variable name, add the \_ character between. E.g.
        \verb|x_1| will be $x_1$ in the document. To add a superscript, add
        \_\_ between the main and the sub. E.g. \verb|x__t| will be $x^t$ in
        the document. The nice thing is that this will not conflict with power
        operations.
    \item[Greek Letters] To insert Greek letters, just write their name, like
        \verb|alpha| for $\alpha$ or \verb|Omega| for $\Omega$. For a full list
        of the letters supported, see \nameref{greek}.
    \item[Accents] To add accents, like dot on top, add \_ and the name of the
        accent. E.g. \verb|V_dot| for $\dot{V}$ or \verb|A_hat| for $\hat{A}$.
        For a full list of accents, see \nameref{accents}.
    \item[Primes] To add prime signs, add \_ and ``[n]prime'', where [n] is the
        number of the primes, up to three, not needed for one. E.g.
        \verb|A_prime| for $A'$ or \verb|A_3prime| for $A'''$. You can refer to
        \nameref{primes}.
    \item[Special] These are simplified versions of some terms. Currently
        supported are \verb|degC| for $^\circ\mathrm{C}$, \verb|degF| for
        $^\circ\mathrm{F}$ and \verb|deg| for $^\circ$. You can refer to
        \nameref{transformed}.
\end{description}

The above can be used to create complicated names like\\
\verb|a_dot_beta_prime__t_tilde| for
\[
    \displaystyle {\dot{a}}_{{\beta}'}^{\tilde{t}}
\]

\subsubsection{Options}
\label{options}

The calculations can have options. Options are a way to customize the
calculations.  They can be inserted after the equation by preceding them with
the \# character, like:

\begin{verbatim}
x = 10 # [option],[option],...
\end{verbatim}

Where the [option] part is an option for that particular line.  As shown above,
multiple options are separated with commas. The available options are:

\begin{description}
    \setlength\itemsep{0pt}
    \item[Unit] You can write units and they will appear next to the number
        whenever the variable is referred. To insert a unit, enter it as a
        correct expression, meaning don't write \verb|W/m^2-K| but write
        \verb|W/(m^2*K)|. Simple evaluation of units is also supported, like
        $\mathrm{10\,m/5\,s=2m/s}$ and cancelling out, like
        $\mathrm{10\,m^2/5\,m=2\,m}$. But the unit that appears might not be
        the intended one as there can be many combinations of the same unit,
        like $\mathrm{Pa = N/m^2 = kg/s^2}$. In that case, you can override it
        with the intended unit. Once the intended unit is given, subsequent
        calculations having an equivalent unit will use that unit, unless they
        are overridden again.

        \begin{verbatim}
        x = 5 # m/s
        \end{verbatim}

    \item[Note] A note is some text after the last step in the calculation. It
        can be handy to say something right after the result. To insert a note,
        precede it with an additional \# character and write what you want. An
        example with a unit and a note

        \begin{verbatim}
        y = 2/x # m^2, #...Ans
        \end{verbatim}

        Where the first \# character signifies the start of options entry and
        the second one signifies the note entry.

    \item[Display or Inline] There are two modes of equations and expressions:
        in-line and display. The in-line equations are those that flow with
        surrounding text in the paragraph, like: $x = y + 2$. But displayed
        equations are those that stand on their own in the middle like:

        \[
            x = y + 2
        \]

        To make the equation in-line, the option is a single \$ sign. For
        display, double the sign, \$\$. The default mode is display. The
        following is an in-line equation.

        \begin{verbatim}
        x = 5 + y #$
        \end{verbatim}

    \item[Vertical or Horizontal] When there is more than one step, the
        subsequent steps are entered below the previous, like
        \[
            \begin{aligned}
                x &= 5 + y\\
                  &= 5 + 5\\
                  &= 10
            \end{aligned}
        \]
    But it might be desired that the steps go horizontally, like
    \[
        x = 5+y = 5+5 = 10
    \]
    To control this, use the hyphen character (-) for horizontal and the pipe
    character (\verb+|+, located above the Enter key, holding Shift) for
    vertical.

    \item[Steps] There are at most three steps to a calculation:
        \begin{enumerate}
            \setlength\itemsep{0pt}
            \item The one with the formula to be evaluated\\ $x = y + z$
            \item The formula with the variables substituted with their
                values\\ $=10+5$
            \item The answer\\ $15$
        \end{enumerate}

        But all of them may not be desired, maybe only the answer is desired.
        To include only some steps, include the step numbers of the desired
        ones. For example, \verb|x = y + z #13| will include steps 1 and 3,
        removing the middle step. \verb|x = y + z #3, m| will only include the
        answer, giving it m as its unit.

    \item[Hide] To hide a calculation but still evaluate it, include the ;
        character in the options. This can be used to set some constants. E.g.
        \verb|R_u = 8.31447 #kJ/(kmol*K), ;| evaluates the equation, with the
        unit, and the constant can be used later, but this equation will not be
        in the document.

\end{description}

\subsection{Text}

Normal paragraphs can be written as text. To avoid mixing the text with the
calculations, \# characters at line beginnings are necessary. And lines with no
empty lines between them are treated as they are in the same paragraph. To
separate paragraphs, use empty lines.  For example, the following will be taken
as two paragraphs

\begin{verbatim}
# Some text in the first paragraph
# and some other text in the same paragraph

# Text in a new paragraph
# and some additional content
\end{verbatim}

An important feature in the text is that values assigned before it can be
referred by using the tag notation.

\begin{verbatim}
x = 10

# since the value of x is #x, ...
\end{verbatim}

\subsection{Displayed Equations and Expressions}

These are equations that will not be evaluated. Since they are not evaluated,
they can be flexible. The left side can be an expression. Also, a long equation
can be broken to a new line. This can be handy to show some procedure.

Example usage:
\begin{verbatim}
#$$ 5*x+9=9*y |=9*sin(10) |=9*0.75 |=128
\end{verbatim}

will come out like:
\[
    \begin{aligned}
        5\,x + 9 &=9\, y\\
                 &=9\, \sin{10}\\
                 &=9\times 0.75\\
                 &=128
    \end{aligned}
\]

There are two options for equations and expressions: in-line and displayed. The
in-line equations are those that flow with surrounding text in the paragraph,
like: $x = y + 2$. But displayed equations are those that stand on their own in
the middle like:

\[
    x = y + 2
\]

To make the in-line equations and expressions flow with the text around them,
insert them between lines supposed to be in the same paragraph, and use a
single \$ sign in the beggining, like so:

\begin{verbatim}
# Some text
#$ x = 5 + 2
# Some other text
\end{verbatim}

\subsection{Calculated Expressions}
\label{expressions}

These are not very much different from the calculated equations except the fact
that they are not equations and thus their result is not retained for later
use. The same options apply to them.  They can be used when explaining
something, like

\begin{verbatim}
# Since the value of #x is less than
15 + x #m,$
# then...
\end{verbatim}

\subsection{Comments}

Comments are a way to enter text that will not be part of the document or the
calculations but are written to explain something in the calculations. Their
existence (or absence thereof) will not have any effect on the operations. To
insert a comment line, begin with two hash characters like:

\begin{verbatim}
c = 5
## this is a comment
v = 10*c
\end{verbatim}

\section{Appendix}

\subsection{Supported Greek Letters}
\label{greek}

\begin{tabular}{ll}
    \toprule
    Letter & Name\\
    \midrule
    $\alpha$      & \verb|alpha|\\
    $\nu$         & \verb|nu|\\
    $\beta$       & \verb|beta|\\
    $\xi$         & \verb|xi|\\
    $\Xi$         & \verb|Xi|\\
    $\gamma$      & \verb|gamma|\\
    $\Gamma$      & \verb|Gamma|\\
    $\delta$      & \verb|delta|\\
    $\Delta$      & \verb|Delta|\\
    $\pi$         & \verb|pi|\\
    $\Pi$         & \verb|Pi|\\
    $\epsilon$    & \verb|epsilon|\\
    $\varepsilon$ & \verb|varepsilon|\\
    $\rho$        & \verb|rho|\\
    $\varrho$     & \verb|varrho|\\
    $\zeta$       & \verb|zeta|\\
    $\sigma$      & \verb|sigma|\\
    $\Sigma$      & \verb|Sigma|\\
    $\eta$        & \verb|eta|\\
    $\tau$        & \verb|tau|\\
    $\theta$      & \verb|theta|\\
    $\vartheta$   & \verb|vartheta|\\
    $\Theta$      & \verb|Theta|\\
    $\upsilon$    & \verb|upsilon|\\
    $\Upsilon$    & \verb|Upsilon|\\
    $\iota$       & \verb|iota|\\
    $\phi$        & \verb|phi|\\
    $\varphi$     & \verb|varphi|\\
    $\Phi$        & \verb|Phi|\\
    $\kappa$      & \verb|kappa|\\
    $\chi$        & \verb|chi|\\
    $\lambda$     & \verb|lambda|\\
    $\Lambda$     & \verb|Lambda|\\
    $\psi$        & \verb|psi|\\
    $\Psi$        & \verb|Psi|\\
    $\mu$         & \verb|mu|\\
    $\omega$      & \verb|omega|\\
    $\Omega$      & \verb|Omega|\\
    \bottomrule
\end{tabular}

\subsection{Supported Accents}
\label{accents}

\begin{tabular}{lll}
    \toprule
    Name & Example & Usage\\
    \midrule
    \verb|hat|    & $\hat{a}$    & \verb|a_hat|\\
    \verb|check|  & $\check{a}$  & \verb|a_check|\\
    \verb|breve|  & $\breve{a}$  & \verb|a_breve|\\
    \verb|acute|  & $\acute{a}$  & \verb|a_acute|\\
    \verb|grave|  & $\grave{a}$  & \verb|a_grave|\\
    \verb|tilde|  & $\tilde{a}$  & \verb|a_tilde|\\
    \verb|bar|    & $\bar{a}$    & \verb|a_bar|\\
    \verb|vec|    & $\vec{a}$    & \verb|a_vec|\\
    \verb|dot|    & $\dot{a}$    & \verb|a_dot|\\
    \verb|ddot|   & $\ddot{a}$   & \verb|a_ddot|\\
    \bottomrule
\end{tabular}

\subsection{Supported Primes}
\label{primes}

\begin{tabular}{lll}
    \toprule
    Name & Example & Usage\\
    \midrule
    \verb|prime|    & $a'$    & \verb|a_prime|\\
    \verb|2prime|  & $a''$  & \verb|a_2prime|\\
    \verb|3prime|  & $a'''$  & \verb|a_3prime|\\
    \bottomrule
\end{tabular}

\subsection{Special Transformed}
\label{transformed}

\begin{tabular}{ll}
    \toprule
    Name & Transformed\\
    \midrule
    \verb|degC|    & $^\circ\mathrm{C}$\\
    \verb|degF|  & $^\circ\mathrm{F}$\\
    \verb|deg|  & $^\circ$\\
    \bottomrule
\end{tabular}


\end{document}
